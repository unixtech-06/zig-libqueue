\documentclass{ltjsarticle}
\usepackage[utf8]{inputenc}
\usepackage{listings}
\usepackage{xcolor}
\usepackage{hyperref}
\usepackage{titling}

\lstdefinestyle{zigstyle}{
  language=C,
  basicstyle=\ttfamily\small,
  keywordstyle=\color{blue}\bfseries,
  commentstyle=\color{green!60!black},
  stringstyle=\color{red},
  numbers=left,
  numberstyle=\tiny\color{gray},
  numbersep=5pt,
  breaklines=true,
  showstringspaces=false,
  frame=single,
  rulecolor=\color{black!30},
  captionpos=b,
  backgroundcolor=\color{black!5},
}

\title{Zigによるキュー(Queue)の完全実装ガイド}
\author{作成者名}
\date{}

\begin{document}

% 表紙
\begin{titlingpage}
\maketitle
\end{titlingpage}

% 目次(新しいページ)
\newpage
\tableofcontents

% 本編(新しいページ)
\newpage

\section{はじめに}

このドキュメントでは、プログラミング言語Zigを使用してキュー(Queue)データ構造を実装する方法を詳細に説明します。C言語の知識がある方を対象としています。

\section{Queueの基本構造}

Zigでは、ジェネリックな型とコンテナを使用してQueueを実装します:

\begin{lstlisting}[style=zigstyle]
pub fn Queue(comptime T: type, comptime Container: type) type {
    return struct {
        const Self = @This();
        c: Container, // underlying container

        // メソッドがここに続きます...
    };
}
\end{lstlisting}

\texttt{T}は要素の型、\texttt{Container}は内部で使用するコンテナ型(例:ArrayList)を表します。

\section{Queueのメソッド}

\subsection{初期化と解放}

\subsubsection{init}
\begin{lstlisting}[style=zigstyle]
pub fn init(allocator: std.mem.Allocator) Self
\end{lstlisting}
Queueを初期化します。C言語の\texttt{malloc}に相当しますが、メモリアロケータを引数に取ります。

\subsubsection{deinit}
\begin{lstlisting}[style=zigstyle]
pub fn deinit(self: *Self) void
\end{lstlisting}
Queueが使用しているメモリを解放します。C言語の\texttt{free}に相当します。

\subsection{要素アクセス}

\subsubsection{front}
\begin{lstlisting}[style=zigstyle]
pub fn front(self: *Self) ?*T
\end{lstlisting}
先頭要素へのポインタを返します。Queueが空の場合は\texttt{null}を返します。

\subsubsection{frontConst}
\begin{lstlisting}[style=zigstyle]
pub fn frontConst(self: *const Self) ?T
\end{lstlisting}
先頭要素の値を返します。Queueが空の場合は\texttt{null}を返します。

\subsubsection{back}
\begin{lstlisting}[style=zigstyle]
pub fn back(self: *Self) ?*T
\end{lstlisting}
末尾要素へのポインタを返します。Queueが空の場合は\texttt{null}を返します。

\subsubsection{backConst}
\begin{lstlisting}[style=zigstyle]
pub fn backConst(self: *const Self) ?T
\end{lstlisting}
末尾要素の値を返します。Queueが空の場合は\texttt{null}を返します。

\subsection{容量}

\subsubsection{empty}
\begin{lstlisting}[style=zigstyle]
pub fn empty(self: *const Self) bool
\end{lstlisting}
Queueが空かどうかを返します。

\subsubsection{size}
\begin{lstlisting}[style=zigstyle]
pub fn size(self: *const Self) usize
\end{lstlisting}
Queueの要素数を返します。

\subsection{要素の追加と削除}

\subsubsection{push}
\begin{lstlisting}[style=zigstyle]
pub fn push(self: *Self, value: T) !void
\end{lstlisting}
Queueの末尾に要素を追加します。メモリ割り当てに失敗した場合はエラーを返します。

\subsubsection{pop}
\begin{lstlisting}[style=zigstyle]
pub fn pop(self: *Self) void
\end{lstlisting}
Queueの先頭要素を削除します。Queueが空の場合は何もしません。

\subsection{その他の操作}

\subsubsection{swap}
\begin{lstlisting}[style=zigstyle]
pub fn swap(self: *Self, other: *Self) void
\end{lstlisting}
二つのQueueの内容を交換します。

\subsubsection{emplace}
\begin{lstlisting}[style=zigstyle]
pub fn emplace(self: *Self, args: anytype) !void
\end{lstlisting}
新しい要素をQueueに直接構築します。C++の\texttt{emplace}に相当します。

\subsection{比較関数}

\subsubsection{isEqual}
\begin{lstlisting}[style=zigstyle]
pub fn isEqual(self: *const Self, other: *const Self) bool
\end{lstlisting}
二つのQueueが等しいかどうかを判断します。

\subsubsection{isLessThan}
\begin{lstlisting}[style=zigstyle]
pub fn isLessThan(self: *const Self, other: *const Self) bool
\end{lstlisting}
一つのQueueが他方より小さいかどうかを判断します。

\subsubsection{isLessThanOrEqual}
\begin{lstlisting}[style=zigstyle]
pub fn isLessThanOrEqual(self: *const Self, other: *const Self) bool
\end{lstlisting}
一つのQueueが他方以下かどうかを判断します。

\subsubsection{isGreaterThan}
\begin{lstlisting}[style=zigstyle]
pub fn isGreaterThan(self: *const Self, other: *const Self) bool
\end{lstlisting}
一つのQueueが他方より大きいかどうかを判断します。

\subsubsection{isGreaterThanOrEqual}
\begin{lstlisting}[style=zigstyle]
pub fn isGreaterThanOrEqual(self: *const Self, other: *const Self) bool
\end{lstlisting}
一つのQueueが他方以上かどうかを判断します。

\subsection{追加機能}

\subsubsection{at}
\begin{lstlisting}[style=zigstyle]
pub fn at(self: *const Self, index: usize) ?T
\end{lstlisting}
指定したインデックスの要素を返します。インデックスが範囲外の場合は\texttt{null}を返します。

\subsubsection{underlying}
\begin{lstlisting}[style=zigstyle]
pub fn underlying(self: *const Self) *const Container
\end{lstlisting}
基礎となるコンテナへの直接アクセスを提供します。

\section{グローバル比較関数}

これらの関数は、C++の演算子オーバーロードに相当する機能を提供します。

\begin{lstlisting}[style=zigstyle]
pub fn eql(comptime T: type, comptime Container: type, 
           a: *const Queue(T, Container), 
           b: *const Queue(T, Container)) bool

pub fn lessThan(comptime T: type, comptime Container: type, 
                a: *const Queue(T, Container), 
                b: *const Queue(T, Container)) bool

pub fn lessThanOrEqual(comptime T: type, comptime Container: type, 
                       a: *const Queue(T, Container), 
                       b: *const Queue(T, Container)) bool

pub fn greaterThan(comptime T: type, comptime Container: type, 
                   a: *const Queue(T, Container), 
                   b: *const Queue(T, Container)) bool

pub fn greaterThanOrEqual(comptime T: type, comptime Container: type, 
                          a: *const Queue(T, Container), 
                          b: *const Queue(T, Container)) bool
\end{lstlisting}

これらの関数は、それぞれQueueの等値比較、小なり比較、以下比較、大なり比較、以上比較を行います。

\section{使用例}

以下は、Queueを使用する簡単な例です:

\begin{lstlisting}[style=zigstyle]
var queue = Queue(i32, std.ArrayList(i32)).init(allocator);
defer queue.deinit();

try queue.push(1);
try queue.push(2);

if (queue.front()) |front| {
    std.debug.print("Front: {}\n", .{front.*});
}

while (!queue.empty()) {
    if (queue.frontConst()) |value| {
        std.debug.print("Popped: {}\n", .{value});
    }
    queue.pop();
}
\end{lstlisting}

この例では、整数型のQueueを作成し、要素を追加し、すべての要素を取り出しています。

\section{まとめ}

このZigによるQueue実装は、C言語の知識を持つ人々にとって理解しやすいものとなっています。メモリ管理が明示的であり、エラー処理が組み込まれているのが特徴です。また、ジェネリックプログラミングを活用することで、様々な型に対応できる柔軟な実装となっています。

\end{document}
